%https://www.uweziegenhagen.de/wp-content/uploads/2016/04/UweZiegenhagen.pdf

\documentclass[a4paper,ngerman,12pt,addpoints]{exam}
\usepackage{babel}
\usepackage[top=3.5cm,headsep=0.5cm,headheight=3cm,left=1.5cm,right=1.5cm]{geometry}
\usepackage[utf8]{inputenc}
\usepackage[T1]{fontenc}
\usepackage{booktabs} % schöne Tabellen
\usepackage{graphicx}
\usepackage{minted}
\usepackage{csquotes} % Anführungszeichen
\usepackage{paralist} % kompakte Aufzählungen
\usepackage{amsmath,textcomp,tikz} 
\usepackage{eso-pic} % Bilder im Hintergrund
\usepackage{mdframed}
\usepackage{tabularx}
\usepackage{xcolor}
\usepackage{colortbl}
\usepackage{tabu}
\usepackage{enumitem}
\usepackage{array}
%\usepackage{helvet}
%\renewcommand{\familydefault}{\sfdefault}
%\fontfamily{phv}\selectfont

\pointpoints{Punkt}{Punkte}
\bonuspointpoints{Bonuspunkt}{Bonuspunkte}
\renewcommand{\solutiontitle}{\noindent\textbf{Lösung:}%
\enspace}
\chqword{Frage} 
\chpgword{Seite}
\chpword{Punkte}
\chbpword{Bonus Punkte}
\chsword{Erreicht}
\chtword{Gesamt}

\hpword{Punkte:} % Punktetabelle
\hsword{Ergebnis:}
\hqword{Aufgabe:}
\htword{Summe:}

\newcommand{\exdate}{1. Februar 2020} % nicht verwenden => extra Deckblatt
\newcommand{\exroom}{HS 1 2 3} % nicht verwenden => extra Deckblatt
\newcommand{\exsubject}{Programmieren in C}
\newcommand{\exsubjectfooter}{\exsubject}
\newcommand{\exsemester}{WS 20/21}
\newcommand{\exrunner}{Prof. Dr. Sven Karol}
\newcommand{\exschool}{Hochschule Merseburg}
\newcommand{\exduration}{60 Minuten}
\newcommand{\exaids}{\textbf{Zulässige Hilfsmittel in nichtdigitaler Form:} eigene Notizen (1 Blatt A4, beidseitig) sowie das Vorlesungsskript (Slides) oder ein Buch zur C Programmierung. Andere Hilfsmittel und insbesondere die Kommunikation mit Anderen sind nicht erlaubt!}

\renewcommand\questionlabel{\textbf{Aufgabe \thequestion.}}

\pagestyle{headandfoot}
\runningheadrule
\firstpageheadrule
\firstpageheader{Name:}{Kennnummer:}{}
\runningheader{Name:}{Kennnummer:}{}
\firstpagefooter{\exsubjectfooter}{Klausur \exsemester}{Seite \thepage\ von \numpages}
\runningfooter{\exsubjectfooter}{Klausur \exsemester}{Seite \thepage\ von \numpages}
\runningfootrule
\firstpagefootrule
\coverfirstpageheader{\small HS Merseburg, Fachbereich Wirtschaftswissenschaften und Informationswissenschaften}{}{\small \exsemester}
\coverfirstpageheadrule

\newmdenv[linecolor=black,backgroundcolor=gray!15,frametitle={Punktverteilung},leftmargin=1cm,rightmargin=1cm]{infobox}
\newcommand{\punkte}[1]{%
\begin{infobox}
    #1 
\end{infobox}}

\newcommand{\excover}{
    \begin{coverpages}
        
        \noindent{}\textbf{\exschool}\\
        \exrunner
        \vspace{1cm}
        %
        \begin{center}
        \Large
            Klausur zur Lehrveranstaltung\\ \textit{\exsubject}
        \end{center}
        %
        \vspace*{1cm}
        %
        \begin{center}
        \begin{tabularx}{450pt}{|X|X|}%
        \hline
        %\rowcolor[gray]{.9}
        %\textbf{Datum:} \exdate & \textbf{Raum:} \exroom \\
        %\hline
        \rowcolor[gray]{.9}
        \multicolumn{2}{|>{\hsize=\dimexpr2\hsize+2\tabcolsep+1\arrayrulewidth\relax}X|}{\textbf{Bearbeitungszeit:} \exduration}\\
        \hline
        \rowcolor[gray]{.9}
        \multicolumn{2}{|>{\hsize=\dimexpr2\hsize+2\tabcolsep+1\arrayrulewidth\relax}X|}{\exaids}\\
        \hline
        \rowcolor[gray]{.9}
        \multicolumn{2}{|>{\hsize=\dimexpr2\hsize+2\tabcolsep+1\arrayrulewidth\relax}X|}{\parbox{0.8\textwidth}{%
        \vspace{2pt}
        \textbf{Sonstige Hinweise:}%
        \vspace{-\topsep}
        \begin{itemize}[leftmargin=*]%
            \setlength{\parskip}{0pt}
            \setlength{\itemsep}{0pt plus 1pt}
            \item Die Klausur besteht aus \numpages{} Seiten.
            \item Tragen Sie Ihren Vor- und Nachnamen, Ihre Matrikelnummer und Ihren Studiengang auf dem Deckblatt ein, bevor Sie mit der Bearbeitung beginnen!
            \item Tragen Sie auf jedem Arbeitsblatt Ihren Namen und Ihre Matrikelnummer in die jeweilige Kopfzeile ein.
            \item Schreiben Sie Ihre Antworten an die entsprechenden Stellen des Aufgabenblatts und nutzen Sie ggf. die leeren Rückseiten!
            %\item Sofern nicht anders angegeben, nehmen Sie die Korrektheit von gegebenem Programm-Code an und ergänzen Sie fehlende Programmteile durch schlüssige Annahmen!
            \item Schreiben Sie deutlich, unkenntliche Antworten werden nicht berücksichtigt!
        \end{itemize}%
        \begin{center}\textbf{Viel Erfolg!}\end{center} 
        }}\\
        \hline
        \end{tabularx}
        \end{center}
        %
        \vspace*{1cm}
        %
        \begin{center}
        \extrarowsep=5pt
        \begin{tabu} to 330pt {| X | X[2] |}%
        \hline
        \textbf{Vorname:}& \\
        \hline
        \textbf{Nachname:}& \\
        \hline
        \textbf{Kennnummer:}& \\
        \hline
        \textbf{Studiengang:}& \\
        \hline
        \end{tabu}
        \end{center}
        %
        \vspace*{1cm}
        %
        \begin{center}
        \gradetable[h][questions] 
        \end{center}
        
        
        \end{coverpages}
}

%https://tex.stackexchange.com/questions/332357/customizing-the-length-of-dotfill
%\newcommand\dotline[2][.5em]{\leavevmode\hbox to #2{\dotfill{#1}\hfill}}

\newcommand{\excoverplus}{
    \setlength{\parindent}{0pt}
    \begin{coverpages}
        \thispagestyle{headandfoot}
        
        \vspace*{0.1cm}
        
        \large
        \begin{center}
         {\renewcommand{\arraystretch}{1.3}
        \begin{tabular}{ll}
        \textbf{Modulklausur:} & \textbf{\exsubject}  \\
        \textbf{Prüfer:} & \textbf{\exrunner} \\
        \textbf{Datum:} & \makebox[4.5cm]{\dotfill}\\
        \end{tabular}
         }
        \end{center}
        \normalsize
        
        \vspace*{0.9cm}
        
        Dieser Klausurteil besteht aus \numpages{} Seiten. Die Vollständigkeit der Aufgabenstellungen wurde von mir zu Beginn der Klausur geprüft und mit meiner Unterschrift bestätigt: 
        
        \vspace*{1cm}
        
        {\renewcommand{\arraystretch}{0.8}
        \noindent\begin{tabularx}{\textwidth}{llllXlX}
        \makebox[2.5cm]{\dotfill} &&  \makebox[2.5cm]{\dotfill}  && \dotfill && \dotfill \\
        Kenn-Nr. && Studiengang && Name, Vorname && Unterschrift \\
        && && des Klausurteilnehmers && des Klausurteilnehmers \\
        && && (Blockschrift) && \\
        \end{tabularx}}
        %\renewcommand{\arraystretch}{1}
        
        \vspace*{0.6cm}
        
        \begin{minipage}{0.7\textwidth}
            \centering
            \gradetable[h][questions]        
        \end{minipage}%
        \begin{minipage}{0.2\textwidth}
        {\renewcommand{\arraystretch}{1.5}% 
            \begin{tabularx}{3cm}{|X|}
                \hline
                \centering Note: \tabularnewline
                \hline
                \rule{0cm}{1.0cm} \tabularnewline
                \hline
            \end{tabularx}}  
        \end{minipage}%
 
        %\begin{tabularx}{\textwidth}{|X|X|X|X|}
        %\hline
        %\centering Punkte & \centering evtl. Bonuspunkte & \centering Gesamtpunkte & \centering evtl. Note \tabularnewline
        %\hline
        %\rule{0cm}{1.0cm} &&& \tabularnewline
        %\hline
        %\end{tabularx}
        
        \vspace*{0.5cm}
        \begin{tabularx}{\textwidth}{lX}
        Erreichbare Punktzahl: & \numpoints{} Punkte\\
        Bearbeitungszeit: & \exduration\\
        Zugelassene Hilfsmittel: & \exaids\\
        \end{tabularx}
        
        \vspace*{0.6cm}
        
        Tragen Sie auf jeder Seite Ihren Namen und Ihre Kennnummer in die Kopfzeile ein!
        \vspace*{0.4cm}

        Anzahl der vom Klausurteilnehmer angefügten Zusatzblätter: \dotfill
        
        \vspace*{0.4cm}
        


        Bitte achten Sie auf eine leserliche Schrift!
        \vspace*{0.4cm}

        \hrulefill
    \end{coverpages}
}